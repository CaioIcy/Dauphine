\chapter[Introdução]{Introdução}

\section{Resumo da Proposta}
% [Descrever o resumo da proposta de desenvolvimento / uso de ferramentas.]
Atualmente, o desenvolvimento de jogos indie tem um grande déficit na área de testes. A proposta deste trabalho é analisar as dificuldades de fazer testes unitários em um jogo. Para isso, será utilizado o jogo Dauphine, desenvolvido na disciplina de Introdução a Jogos Eletrônicos da UnB-FGA.

\section{Metodologia}

Para a elaboração dos testes unitários, a ferramenta Google Test está sendo utilizada. Após determinadas baterias de testes, a qualidade do código-fonte seria avaliada utilizando o Analizo para análise estática, para poder observar os benefícios dos testes espelhadas nas métricas coletadas. Também está sendo utilizado o Travis CI para integração contínua do projeto. Após a elaboração dos testes e devidamente registradas as dificuldades do processo, serão entrevistados desenvolvedores profissionais da área para saber como eles resolvem esses problemas e outros possíveis probemas ou até mesmo se eles chegam neles.

\section{Dificuldades Encontradas}
% [Listar as dificuldades encontradas, de qualquer natureza, durante o desenvolvimento do trabalho.]
Algumas dificuldades foram encontradas durante a execução deste trabalho nessa sessão são elencadas as principais.

\subsection{Cobertura}
% Ao executar o trabalho proposto, percebeu-se que realizar testes unitários em jogos eletrônicos é uma tarefa difícil. A dependência de todos os módulos é muito grande. Durante o período da disciplina, foi observado que a capacidade do jogo ser testado está diretamente ligado a qualidade do código originalmente desenvolvido.

Não foram encontradas ferramentas viáveis de serem utilizadas que gerassem um relatório de cobertura de testes unitários. Dentre as ferramentas investigadas seguem as principais.

\begin{itemize}
\item \textbf{Cpputest:} (\textit{Open Source}) Não mede cobertura, é só um \textit{framework} de testes unitários escritos em C++ mas usados para projetos C/C++.
\item \textbf{BullseyeCoverage:} (Proprietário) É um analisador de cobertura usado para melhorar a qualidade do software em sistemas vitais como aplicações de empresas, controle de indústria, médico, automotiva, comunicações, aeroespacial e defesa.
\item \textbf{Semantic Designs C++:}  (Proprietário) Provê cobertura de testes para linguagens procedurais como C, C++, Java, C\#, PHP e diversas outras.
\item \textbf{C++ Test Coverage Tool:}  (Proprietário)
\item \textbf{Squish Coco:} (Proprietário) Usa instrumentação de código-fonte para analisar a aplicação. Nenhuma mudança no código é necessária.
\end{itemize}

As ferramentas acima, por serem proprietárias e algumas obsoletas, não puderam ser utilizadas no projeto pelo alto custo de suas licenças.

\subsection {Qualidade}
A má qualidade do código do jogo Dauphine impossibilitou que os testes fossem implementados da maneira planejada. Por exemplo, certos acoplamentos são inevitáveis na construção da \textit{engine} do jogo.
% Durante o período da disciplina, foi observado que a capacidade do jogo ser testado está diretamente ligado a qualidade do código originalmente desenvolvido.
\subsection {Referencial Teórico}
A escassez de material de cunho acadêmico em relação ao desenvolvimento de jogos foi, apesar de esperado, prejudicial.

