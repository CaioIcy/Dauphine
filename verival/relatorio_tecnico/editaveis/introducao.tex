\chapter[Introdução]{Introdução}

\section{Resumo da Proposta}
% [Descrever o resumo da proposta de desenvolvimento / uso de ferramentas.]
Atualmente, o desenvolvimento de jogos indie tem um grande déficit na área de testes. A proposta deste trabalho é analisar as dificuldades de fazer testes unitários em um jogo e seus benefícios para a construção do mesmo, mostrando também quais partes de um jogo valem mais a pena serem testadas. Para isso, será utilizado o jogo Dauphine, desenvolvido na disciplina de Introdução a Jogos Digitais da UnB-FGA. Para a elaboração dos testes unitários, a ferramenta Google Test está sendo utilizada. Após determinadas baterias de testes, a qualidade do código-fonte está sendo avaliada utilizando o Analizo para análise estática, para pode observar os benefícios dos testes espelhadas nas métricas coletadas. Também está sendo utilizada o Travis CI para integração contínua do jogo. Após a elaboração dos testes e devidamente registradas as dificuldades do processo, serão entrevistados desenvolvedores indie para saber como eles resolvem esses problemas e outros possíveis probemas ou até mesmo se eles chegam neles.

\section{Dificuldades Encontradas}
% [Listar as dificuldades encontradas, de qualquer natureza, durante o desenvolvimento do trabalho.]
As dificuldades encontradas até agora no trabalho foram
\begin{itemize}
\item Definição de priorização de quais módulos do jogo serem testados;
\item Problemas com a ferramenta de análise estática (Analizo x Mezuro);
\item Problemas na formalização dos objetivos do trabalhos;
\item Problemas com a coleta da cobertura de código
\end{itemize}
