\chapter[Resultados Alcançados]{Resultados Alcançados}
% [Em relação aos objetivos planejados, faça uma análise crítica dos resultados que foram alcançados.
% 	Todos os objetivos foram alcançados?
% 	Quais foram parcialmente alcançados? Por que?
% 	O produto é relevante? Por que?
% 	Proponha trabalhos futuros.
% ]

Nesta seção, serão mostrados os resultados obtidos durante a execução do trabalho, bem como as dificuldades encontradas para a realização do mesmo.

\section{Resultados do Objetivo Principal}

% Entender as dificuldades de testes unitários especificamente na área de desenvolvimento de jogos.
No decorrer do desenvolvimento do trabalho, foi concluído que testes unitários em jogos, apesar de necessários, são difíceis de serem realizados e demandam um tempo grande, muitas vezes comprometendo o tempo de entrega do jogo. Por isso, os testes unitários infelizmente costumam ser negligenciados pelas equipes de desenvolvimento.

\section{Resultados dos Objetivos Específicos}

Apesar da ferramenta de integração contínua (\textit{Travis CI}) estar devidamente configurada, seu uso não foi possível devido ao fato da equipe ter falhado em realizar os testes unitários.

A proposta de acompanhamento de evolução da qualidade através da ferramenta \textit{Analizo} também foi comprometido devido ao fato de que não houveram refatorações originadas de testes unitários falhando, uma vez que estes não existiram.

Foram estudadas diversas ferramentas para obter a cobertura, porém nenhuma das estudadas foram capazes de realizar a instrumentação do código para que pudesse ser obtido o valor da cobertura.

Devido às dificuldades descritas anteriormente, não foi possível obter uma cobertura de 60\% do código.

Ao realizar um questionário com desenvolvedores profissionais da área, foi observado que, além dos que não sabem nem o que é um teste automatizado, a maioria não implementa testes unitários em seus códigos. Assim, a equipe não pode avaliar como os desenvolvedores profissionais resolviam os problemas encontrados, pois os mesmos nem ao menos chegavam nessas dificuldades.

\chapter[Conclusão]{Conclusão}

Testes unitários são importantes para o desenvolvimento de qualquer software, para facilitar a manutenção sendo ela corretiva ou evolutiva, para que o desenvolvedor possa com certo nível de confiabilidade garantir que ao desenvolver novas funcionalidades as outras continuem funcionando normalmente, ou ao refatorar o código não se perca o controle da funcionalidade.

No contexto de jogos eletrônicos as motivações explicitadas se mantém, porém a complexidade de realizar tais testes pode ser agravada devido ao alto acoplamento inerente à arquitetura de jogos \cite{gregory} e a falta de frameworks específicos para testes em jogos. Isso eleva o tempo de implementação dos teste.

Apesar dos desenvolvedores profissionais de jogos eletrônicos terem uma bagagem teórica sobre testes unitários, poucos desenvolvedores alegam realizar esse tipo de teste, principalmente pelo motivo citado anteriormente. O reflexo disso é a quantidade de \textit{bugs} que passam para a versão final do jogo.

Assim como qualquer outro tipo de \textit{software}, a qualidade do código-fonte influencia na facilidade de testar. Códigos complexos são mais difíceis de testar.