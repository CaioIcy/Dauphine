\chapter[Introdução]{Introdução}

\section{Contexto}

Os testes estão no coração da engenharia de software, porém no desenvolvimento de jogos nem sempre eles são trabalhados. A área de jogos é extremamente multidisciplinar e isso talvez leve o desenvolvedor a relevar os testes, achando que não são necessários. Apesar de tudo, um jogo em seu núcleo é apenas um software e os testes são extremamente necessários.

Neste projeto será usado o jogo Dauphine, desenvolvido na matéria de graduação Introdução aos Jogos Eletrônicos da Universidade de Brasília, para ser investigado quais são os motivos desta falta de cuidado, bem como as suas dificuldades.

\section{Formulação do Problema}

Uma engine de jogo é algo extremamente complexo de ser construído, pois certos acoplamentos são inevitáveis \cite{gregory}. Esta complexidade impacta diretamente na hora de testar o jogo. Usando o projeto Dauphine, será analisado tópicos relacionados a teste de software desde um nível técnico, como ferramentas de teste, até um nível mais abstrato, como por que razão não existe uma cultura de teste na área de jogos, principalmente com desenvolvedores indies.

\section{Objetivos}

\subsection{Objetivo Geral}
Entender as dificuldades de testes unitários especificamente na área de desenvolvimento de jogos.

\subsection{Objetivos Específicos}
\begin{itemize}
\item Identificar ferramentas para testes unitários em C++;
\item Levantar dificuldades na implementação dos testes unitários em jogos;
\item Analisar material teórico sobre testes unitários em jogos;
\item Conhecer a opinião sobre testes unitários de desenvolvedores indies de jogos;
\end{itemize}

\section{Justificativas}

A falta de material de referência sobre teste, especificamente na área de desenvolvimento de jogos, já é um alerta em si para a importância da realização deste projeto.

Com a área de jogos crescendo cada vez mais no mercado, é fundamental que exista um arcabouço teórico por trás validando a parte técnica para que os desenvolvedores não se encurralem em um caminho sem saída.
