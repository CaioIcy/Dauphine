\chapter[Itens de Testes]{Itens de Testes}
% [É impossível testar 100 porcento do software. Desta forma, é essencial para o sucesso do projeto selecionar os produtos e componentes de software mais apropriados e importantes para o teste. Os itens escolhidos para teste representarão um equilíbrio entre o custo, o risco e a necessidade de verificá-los.

% Listar os produtos ou componentes de produto que serão testados, descrevendo:
% Item de Teste: Nome do produto ou componente a ser testado;
% Potencial Modo de Falha: Descrição da(s) falha(s) normalmente verificada(s) no produto ou componentes a serem testados;
% Potencial Efeito da Falha: Descrição das consequências da falha anteriormente descrita no sistema e/ou projeto;
% Potencial Causa da Falha: Descrição das causas principais da falha anteriormente descrita;
% Níveis: Descrever os níveis de teste que serão aplicados ao produto ou componente de software:
% Unidade (UNI);
% Integração (INT);
% Sistema (SIS);
% Aceitação (ACE).
% Tipo: Tipo de teste a ser aplicado no produto ou componente de produto.]

Os itens de teste serão divididos em módulos da arquitetura por trás do Dauphine (audio, core, engine, graphics, input, util). 

\begin{table}[h]
\centering
\begin{tabular}{|p{1.3cm}|p{3.4cm}|p{3.4cm}|p{3.4cm}|p{1.2cm}|p{2.5cm}|}
\hline
\textbf{Item de Teste} & \textbf{Potencial Modo de Falha} & \textbf{Potencial Efeito da Falha} & \textbf{Potencial Causa da Falha} & \textbf{Níveis} & \textbf{Tipo} \\ \hline
audio	& Arquivo de áudio não carregado & O jogo não toca os áudios necessários & Falha na escrita do nome do arquivo de áudio no código ou nos scripts, ou arquivo de áudio não se encontra no local desejado & UNI & Confiabilidade \\ \hline
core	& Sistemas da libSDL não conseguem ser ligados & O programa se fecha & Programa mal configurado ou bibliotecas não configuradas & UNI & Confiabilidade \\ \hline
engine	& Máquina de estados finita (FSM) dos estados do jogo não funcionam & Jogo não roda na ordem desejada & Código duplicado e confuso & UNI & Desempenho \\ \hline
graphics& Falha na animação dos sprites & Propagação do erro para as outras animações & Variável global que controla o tempo de animação & UNI & Confiabilidade \\ \hline
input	& Falha no reconhecimento de um joystick & O único meio de controle se torna o teclado, caso disponível & API confusa ou joystick não configurado corretamente & UNI & Confiabilidade \\ \hline
util	& Scripts LUA não carregados & O programa se fecha & Problemas com API do LUA & UNI & Confiabilidade \\ \hline
\end{tabular}
\caption{Os itens de teste.}
\label{itens_teste}
\end{table}
