\chapter[Produtos, Atividades e Cronograma]{Produtos, Atividades e Cronograma}

\section{Resumo da proposta}
	Este trabalho pretende analisar a verificação da qualidade no desenvolvimento de jogos eletrônicos a partir da analise estática de código e do desenvolvimento de testes unitários tendo em vista a definição do grau de dificuldade e efetividade do controle de qualidade de código no processo de produção de jogos eletrônicos.
	Determinar não como testar, que segundo \citeonline{artoftest} testar é o processo de executar um programa com a intenção de encontrar erros, mas principalmente o que deve ser testado, dado que no contexto as mudanças acontecem de forma acentuada uma cobertura total aumentaria o trabalho de testar tais mudanças.
\section{Estrutura Analítica do Projeto}
\section{Lista de Software}
\begin{itemize}
\item{Linguagem de programação:} \textbf{C++}
Sua alta performance e qualidade, segue na liderança disparada em produção de jogos.
\item{Compilador:} \textbf{GNU Compiler Collection (gcc)}
Confiável e com bom suporte ao padrão C++11.
\item{Controle de versão de código e documentação:} \textbf{Git (GitHub)}
Na liderança popular entre os forges de Git por sua qualidade e poder de socialização. O repositório pode ser acessado em $github.com/CaioIcy/Dauphine$.
\item{Editor de texto:} \textbf{Sublime Text}
Preferido pelos desenvolvedores da equipe.
\item{Gerador de documentação:} \textbf{Doxygen}
Um gerador de documentação excelente, e compátível com C++. A documentação do código será hospedada \textit{online} em uma \textit{GitHub page}.
\item{Sistema operacional de desenvolvimento:} \textbf{Linux Debian Based}
\end{itemize}
\section{Cronograma de Atividades}
