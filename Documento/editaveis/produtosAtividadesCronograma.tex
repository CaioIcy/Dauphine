\chapter[Produtos, Atividades e Cronograma]{Produtos, Atividades e Cronograma}

\section{Resumo da proposta}
	Este trabalho pretende analisar a verificação da qualidade no desenvolvimento de jogos eletrônicos a partir da analise estática de código e do desenvolvimento de testes unitários tendo em vista a definição do grau de dificuldade e efetividade do controle de qualidade de código no processo de produção de jogos eletrônicos.
	Determinar não como testar, que segundo \citeonline{artoftest} testar é o processo de executar um programa com a intenção de encontrar erros, mas principalmente o que deve ser testado, dado que no contexto as mudanças acontecem de forma acentuada uma cobertura total aumentaria o trabalho de testar tais mudanças.
\section{Estrutura Analítica do Projeto}
\section{Lista de Software}
\section{Cronograma de Atividades}
